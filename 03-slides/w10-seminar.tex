\documentclass[12pt,english,dvipsnames,aspectratio=169,handout]{beamer}\usepackage[]{graphicx}\usepackage[]{xcolor}
% maxwidth is the original width if it is less than linewidth
% otherwise use linewidth (to make sure the graphics do not exceed the margin)
\makeatletter
\def\maxwidth{ %
  \ifdim\Gin@nat@width>\linewidth
    \linewidth
  \else
    \Gin@nat@width
  \fi
}
\makeatother

\definecolor{fgcolor}{rgb}{0.345, 0.345, 0.345}
\newcommand{\hlnum}[1]{\textcolor[rgb]{0.686,0.059,0.569}{#1}}%
\newcommand{\hlstr}[1]{\textcolor[rgb]{0.192,0.494,0.8}{#1}}%
\newcommand{\hlcom}[1]{\textcolor[rgb]{0.678,0.584,0.686}{\textit{#1}}}%
\newcommand{\hlopt}[1]{\textcolor[rgb]{0,0,0}{#1}}%
\newcommand{\hlstd}[1]{\textcolor[rgb]{0.345,0.345,0.345}{#1}}%
\newcommand{\hlkwa}[1]{\textcolor[rgb]{0.161,0.373,0.58}{\textbf{#1}}}%
\newcommand{\hlkwb}[1]{\textcolor[rgb]{0.69,0.353,0.396}{#1}}%
\newcommand{\hlkwc}[1]{\textcolor[rgb]{0.333,0.667,0.333}{#1}}%
\newcommand{\hlkwd}[1]{\textcolor[rgb]{0.737,0.353,0.396}{\textbf{#1}}}%
\let\hlipl\hlkwb

\usepackage{framed}
\makeatletter
\newenvironment{kframe}{%
 \def\at@end@of@kframe{}%
 \ifinner\ifhmode%
  \def\at@end@of@kframe{\end{minipage}}%
  \begin{minipage}{\columnwidth}%
 \fi\fi%
 \def\FrameCommand##1{\hskip\@totalleftmargin \hskip-\fboxsep
 \colorbox{shadecolor}{##1}\hskip-\fboxsep
     % There is no \\@totalrightmargin, so:
     \hskip-\linewidth \hskip-\@totalleftmargin \hskip\columnwidth}%
 \MakeFramed {\advance\hsize-\width
   \@totalleftmargin\z@ \linewidth\hsize
   \@setminipage}}%
 {\par\unskip\endMakeFramed%
 \at@end@of@kframe}
\makeatother

\definecolor{shadecolor}{rgb}{.97, .97, .97}
\definecolor{messagecolor}{rgb}{0, 0, 0}
\definecolor{warningcolor}{rgb}{1, 0, 1}
\definecolor{errorcolor}{rgb}{1, 0, 0}
\newenvironment{knitrout}{}{} % an empty environment to be redefined in TeX

\usepackage{alltt}
\usepackage{fontspec}
\setsansfont[Mapping=tex-text]{Fira Sans}
\setcounter{secnumdepth}{4}
\setcounter{tocdepth}{4}
\usepackage[normalem]{ulem}
\usepackage[T1]{fontenc}
\usepackage{dcolumn}
\usepackage{booktabs}
\usepackage{bm}
\usepackage{setspace}
\makeatletter
\usetheme{metropolis}
\setbeamertemplate{frame footer}{Bosancianu | Schaub | Hertie School}
\setbeamerfont{page number in head/foot}{size=\tiny}
\setbeamercolor{footline}{fg=gray}
\usepackage{xcolor}
\setbeamercovered{transparent}
\usepackage{tikz}
\usetikzlibrary{arrows, positioning,fit,shapes.misc}
\usepackage[labelformat=empty]{caption}
% For table captions in Beamer
\usepackage[sectionbib]{apacite}
\renewcommand{\bibliographytypesize}{\footnotesize}
\makeatletter
\let\st@rtbibsection\@bibnewpage
\let\st@rtbibchapter\@bibnewpage
\makeatother
\usepackage{amsmath, mathtools}
\usepackage{xunicode}
\usepackage{hyperref}
\graphicspath{{./figures/}} 
% Defines a checkmark
\def\checkmark{\tikz\fill[scale=0.4,color=orange](0,.35) -- (.25,0) -- (1,.7) -- (.25,.15) -- cycle;}
% Code for circles in Table cells
\newcounter{nodemarkers}
\newcommand\circletext[1]{%
    \tikz[overlay,remember picture] 
        \node (marker-\arabic{nodemarkers}-a) at (0,1.5ex) {};%
    #1%
    \tikz[overlay,remember picture]
        \node (marker-\arabic{nodemarkers}-b) at (0,0){};%
    \tikz[overlay,remember picture,inner sep=2pt]
        \node[draw,ellipse,fit=(marker-\arabic{nodemarkers}-a.center) (marker-\arabic{nodemarkers}-b.center)] {};%
    \stepcounter{nodemarkers}%
}
% wide itemize and enumerate
\newenvironment{wideitemize}{\itemize\addtolength{\itemsep}{.3em}}{\enditemize}
\newenvironment{wideenumerate}{\enumerate\addtolength{\itemsep}{.3em}}{\endenumerate}
% boxes
\def\boxitorange#1{%
  \smash{\color{orange}\fboxrule=1pt\relax\fboxsep=2pt\relax%
  \llap{\rlap{\fbox{\vphantom{0}\makebox[#1]{}}}~}}\ignorespaces
}
\def\boxitblue#1{%
  \smash{\color{blue}\fboxrule=1pt\relax\fboxsep=2pt\relax%
  \llap{\rlap{\fbox{\vphantom{0}\makebox[#1]{}}}~}}\ignorespaces
}
\newcommand{\indep}{\perp \!\!\!\! \perp}
\setbeamertemplate{itemize items}{\checkmark}
\usepackage{multirow}
\hypersetup{pdfauthor={Bosancianu and Schaub},
	pdftitle={Statistical Modeling and Causal Inference with R},
	pdfsubject={Week 10: Moderation and heterogeneous effects},
	pdfkeywords={Berlin, Hertie, 2020, week 8}}
\title{\textsc{Statistical Modeling and Causal Inference with R}}
\subtitle{Week 10: Moderation and heterogeneous effects}
\date{November 16, 2020}
\author{Manuel Bosancianu \hfill Max Schaub}
\institute{Hertie School of Governance}
\IfFileExists{upquote.sty}{\usepackage{upquote}}{}
\begin{document}
\maketitle


\begin{frame}
	\frametitle{Lecture Q\&A}
	\begin{itemize} \large
		\item Open Q\&A 
		\item Remarks on assignments
		\item Quiz interpreting regression coefficients
	\end{itemize}
\end{frame}


\begin{frame}
	\frametitle{Remarks on assignments}
\footnotesize

	\begin{enumerate}
		\item Don't leave working on the assignment until shortly before the deadline -- this makes it impossible to get help; when running into serious problems (e.g.\ getting stuck for a couple of hours), seek help in the group or from the TAs
		\item Find a good `balance-table' routine from the tutorials by Adelaida and Sebastian, and use it consistently -- balance tables are very useful (and will be asked for in the exam)
	  \item Do read the question wording very carefully -- often, points are lost because you do something that's not been asked for.
		\item Technical note: Turn off scientific notification in R since this seems to lead to confusion; turn it off (options(scipen=999))
	\end{enumerate}
\end{frame}



\begin{frame}
	\frametitle{Quiz interpreting regression coefficients}
\footnotesize

  What you need:
	\begin{itemize}
		\item A calculator (or being quick at mental arithmetic)
		\item Knowledge of the lecture slides
	\end{itemize}

  Instructions:
	\begin{itemize}
		\item Calculate the response
		\item Post your answer in the poll at \url{https://www.menti.com/tjphg6xova}
	\end{itemize}

\end{frame}



\begin{frame}
	\frametitle{Quiz interpreting regression coefficients}
\footnotesize

You are estimating the effect of a mentoring program on income. 
  
For the evaluation 185 individuals were randomly assigned to a treatment group that received mentoring, while 185 were assigned to a control group that did not. 

You now have data on their incomes 15 months after the intervention. In order to estimate the effect, you are using OLS regression models. 

You start with the simple model in the form:

\normalsize\vspace{3mm}
\textcolor{orange}{Model 1}: ${Y}_i = \beta_0 + \beta_1 {D}_i + \mu_i$
\footnotesize

Where $Y_i$ stands for program participants' income and $D_i$ for the mentoring program.

\end{frame}


\begin{frame}
	\frametitle{Quiz interpreting regression coefficients}
\footnotesize
The results are shown here:

\centering
\tiny
Model: ${Y}_i = \beta_0 + \beta_1 {D}_i + \mu_i$

\scriptsize
\begin{tabular}{l*{1}{c}}
\toprule
          &\multicolumn{1}{c}{Income}\\
\midrule
Mentoring     &     2213\textsuperscript{**}\\
          &    (717)        \\
Intercept    &     4574\textsuperscript{**}\\
          &    (507)        \\
\midrule
N         &      370        \\
\bottomrule
\multicolumn{2}{l}{Standard errors in parentheses. \textsuperscript{†} \(p<0.1\), \textsuperscript{*} \(p<0.05\), \textsuperscript{**} \(p<0.01\)}\\
\end{tabular}

\footnotesize\flushleft

\textcolor{orange}{Q1: What is the treatment effect, i.e.\ the partial effect $\frac{\partial Y_i}{\partial {D}_i}$ of the mentoring program?}\\
\textcolor{blue}{Answer: $\beta_1 = 2213$}

\end{frame}


\begin{frame}
	\frametitle{Quiz interpreting regression coefficients}
\centering
\tiny
Model: ${Y}_i = \beta_0 + \beta_1 {D}_i + \mu_i$

\scriptsize
\begin{tabular}{l*{1}{c}}
\toprule
          &\multicolumn{1}{c}{Income}\\
\midrule
Mentoring     &     2213\textsuperscript{**}\\
          &    (717)        \\
Intercept    &     4574\textsuperscript{**}\\
          &    (507)        \\
\midrule
N         &      370        \\
\bottomrule
\multicolumn{2}{l}{Standard errors in parentheses. \textsuperscript{†} \(p<0.1\), \textsuperscript{*} \(p<0.05\), \textsuperscript{**} \(p<0.01\)}\\
\end{tabular}
\footnotesize\flushleft

\textcolor{orange}{Q2: What is the income of those not in the mentoring program $E[Y_i|D_i=0]$?}\\
\textcolor{blue}{Answer: $\beta_0 = 4574$}

\end{frame}



\begin{frame}
	\frametitle{Quiz interpreting regression coefficients}
\centering
\tiny
Model: ${Y}_i = \beta_0 + \beta_1 {D}_i + \mu_i$

\scriptsize
\begin{tabular}{l*{1}{c}}
\toprule
          &\multicolumn{1}{c}{Income}\\
\midrule
Mentoring     &     2213\textsuperscript{**}\\
          &    (717)        \\
Intercept    &     4574\textsuperscript{**}\\
          &    (507)        \\
\midrule
N         &      370        \\
\bottomrule
\multicolumn{2}{l}{Standard errors in parentheses. \textsuperscript{†} \(p<0.1\), \textsuperscript{*} \(p<0.05\), \textsuperscript{**} \(p<0.01\)}\\
\end{tabular}
\footnotesize\flushleft

\textcolor{orange}{Q3: What is the average income for the whole sample $E[Y_i]$?}\\
\textcolor{blue}{Answer: Weighted average $\frac{185}{370}\times \beta_0 + \frac{185}{370}\times (\beta_0 + \beta_1) = 0.5*4574+0.5*(4574 + 2213)=5680.5$}

\end{frame}




\begin{frame}
	\frametitle{Quiz interpreting regression coefficients}
\footnotesize

Now to improve the efficiency of your estimates, you add an indicator recording whether the participants have a Masters degree. 

This is the indicator U that takes the value 1 for those with a Masters degree (87/370), and 0 for those without (283/370).

This means your second model takes the form:

\normalsize\vspace{3mm}
\textcolor{orange}{Model 2}: ${Y}_i = \beta_0 + \beta_1 D_i + \beta_2 U_i + \epsilon_i$
\footnotesize

\vspace{25mm}

\end{frame}



\begin{frame}
	\frametitle{Quiz interpreting regression coefficients}
\centering
\tiny
Model: ${Y}_i = \beta_0 + \beta_1 D_i + \beta_2 U_i + \epsilon_i$

\scriptsize
\centering
\begin{tabular}{l*{1}{c}}
\toprule
          &\multicolumn{1}{c}{Income}\\
\midrule
Mentoring     &     1962\textsuperscript{**}\\
          &    (718)        \\
Masters degree    &     2216\textsuperscript{**}\\
          &    (847)        \\
Intercept    &     4178\textsuperscript{**}\\
          &    (525)        \\
\midrule
N         &      370        \\
\bottomrule
\multicolumn{2}{l}{Standard errors in parentheses. \textsuperscript{†} \(p<0.1\), \textsuperscript{*} \(p<0.05\), \textsuperscript{**} \(p<0.01\)}\\
\end{tabular}
\footnotesize\flushleft

\textcolor{orange}{Q4: What is the income of those in the mentoring program but without a Masters degree $E[Y_i|D_i=1,U_i=0]$?}\\
\textcolor{blue}{Answer: $\beta0 + \beta1 = 4178+1962 = 6140$}

\end{frame}



\begin{frame}
	\frametitle{Quiz interpreting regression coefficients}
\footnotesize

You suspect that the mentoring program has a different effect for those with a Masters degree vs. those without. You therefore estimate an interaction model in the form:

\normalsize\vspace{3mm}
\textcolor{orange}{Model 3}: ${Y}_i = \beta_0 + \beta_1 D_i + \beta_2 U_i + \beta_3 D_i\times U_i + \mu_i$
\footnotesize

\vspace{25mm}

\end{frame}



\begin{frame}
	\frametitle{Quiz interpreting regression coefficients}
\centering
\tiny
Model: ${Y}_i = \beta_0 + \beta_1 D_i + \beta_2 U_i + \beta_3 D_i\times U_i + \mu_i$

\scriptsize
\begin{tabular}{l*{1}{c}}
\toprule
          &\multicolumn{1}{c}{Income}\\
\midrule
Mentoring     &     1048        \\
          &    (811)        \\
Masters degree    &     -156        \\
          &   (1306)        \\
Mentoring $\times$ Masters degree &     4053\textsuperscript{*} \\
          &   (1708)        \\
Intercept    &     4601\textsuperscript{**}\\
          &    (552)        \\
\midrule
N         &      370        \\
\bottomrule
\multicolumn{2}{l}{Standard errors in parentheses. \textsuperscript{†} \(p<0.1\), \textsuperscript{*} \(p<0.05\), \textsuperscript{**} \(p<0.01\)}\\
\end{tabular}
\footnotesize\flushleft

\textcolor{orange}{Q5: What is the effect of the treatment for those without a Masters degree $\frac{\partial Y_i}{\partial D_i, U_i=0}$?} \textcolor{blue}{Answer: $\beta1  = 1048$}

\end{frame}



\begin{frame}
	\frametitle{Quiz interpreting regression coefficients}
\centering
\tiny
Model: ${Y}_i = \beta_0 + \beta_1 D_i + \beta_2 U_i + \beta_3 D_i\times U_i + \mu_i$

\scriptsize
\begin{tabular}{l*{1}{c}}
\toprule
          &\multicolumn{1}{c}{Income}\\
\midrule
Mentoring     &     1048        \\
          &    (811)        \\
Masters degree    &     -156        \\
          &   (1306)        \\
Mentoring $\times$ Masters degree &     4053\textsuperscript{*} \\
          &   (1708)        \\
Intercept    &     4601\textsuperscript{**}\\
          &    (552)        \\
\midrule
N         &      370        \\
\bottomrule
\multicolumn{2}{l}{Standard errors in parentheses. \textsuperscript{†} \(p<0.1\), \textsuperscript{*} \(p<0.05\), \textsuperscript{**} \(p<0.01\)}\\
\end{tabular}
\scriptsize\flushleft

\textcolor{orange}{Q6: What is the effect of the treatment for those \textbf{with} a Masters degree $\frac{\partial Y_i}{\partial D_i, U_i=1}$? And, Q7, what is the difference between the treatment effects?}\\
\textcolor{blue}{Answer Q6: $\beta1 + \beta3 = 1048 + 4053 =5101$, Answer Q7: $\beta3 = 4053, p<0.01$ as per regression table.}

\end{frame}



\begin{frame}
	\frametitle{Quiz interpreting regression coefficients}
\centering
\tiny
Model: ${Y}_i = \beta_0 + \beta_1 D_i + \beta_2 U_i + \beta_3 D_i\times U_i + \mu_i$

\scriptsize
\begin{tabular}{l*{1}{c}}
\toprule
          &\multicolumn{1}{c}{Income}\\
\midrule
Mentoring     &     1048        \\
          &    (811)        \\
Masters degree    &     -156        \\
          &   (1306)        \\
Mentoring $\times$ Masters degree &    4053\textsuperscript{*} \\
          &   (1708)        \\
Intercept    &     4601\textsuperscript{**}\\
          &    (552)        \\
\midrule
N         &      370        \\
\bottomrule
\multicolumn{2}{l}{Standard errors in parentheses. \textsuperscript{†} \(p<0.1\), \textsuperscript{*} \(p<0.05\), \textsuperscript{**} \(p<0.01\)}\\
\end{tabular}

\tiny{Remember that 87/370 participants have a Masters degree, and 283/370 don't.}
\scriptsize\flushleft

\textcolor{orange}{Q8: What is the \textbf{overall} effect of the mentoring program, i.e.\ the partial effect $\frac{\partial Y_i}{\partial D_i}$?}\\
\textcolor{blue}{Answer: Weighted average $\frac{283}{370}\beta_1 + \frac{87}{370}\times (\beta_1 + \beta_3) = 1048*(283/370)+5101*(87/370)=2001$}

\end{frame}


% END
\begin{frame}
\begin{center}
    \LARGE Thank you for watching, and see you next Monday!
\end{center}
\end{frame}

\end{document}
