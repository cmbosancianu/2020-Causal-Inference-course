\documentclass[12pt,english,dvipsnames,aspectratio=169,handout]{beamer}\usepackage[]{graphicx}\usepackage[]{xcolor}
% maxwidth is the original width if it is less than linewidth
% otherwise use linewidth (to make sure the graphics do not exceed the margin)
\makeatletter
\def\maxwidth{ %
  \ifdim\Gin@nat@width>\linewidth
    \linewidth
  \else
    \Gin@nat@width
  \fi
}
\makeatother

\definecolor{fgcolor}{rgb}{0.345, 0.345, 0.345}
\newcommand{\hlnum}[1]{\textcolor[rgb]{0.686,0.059,0.569}{#1}}%
\newcommand{\hlstr}[1]{\textcolor[rgb]{0.192,0.494,0.8}{#1}}%
\newcommand{\hlcom}[1]{\textcolor[rgb]{0.678,0.584,0.686}{\textit{#1}}}%
\newcommand{\hlopt}[1]{\textcolor[rgb]{0,0,0}{#1}}%
\newcommand{\hlstd}[1]{\textcolor[rgb]{0.345,0.345,0.345}{#1}}%
\newcommand{\hlkwa}[1]{\textcolor[rgb]{0.161,0.373,0.58}{\textbf{#1}}}%
\newcommand{\hlkwb}[1]{\textcolor[rgb]{0.69,0.353,0.396}{#1}}%
\newcommand{\hlkwc}[1]{\textcolor[rgb]{0.333,0.667,0.333}{#1}}%
\newcommand{\hlkwd}[1]{\textcolor[rgb]{0.737,0.353,0.396}{\textbf{#1}}}%
\let\hlipl\hlkwb

\usepackage{framed}
\makeatletter
\newenvironment{kframe}{%
 \def\at@end@of@kframe{}%
 \ifinner\ifhmode%
  \def\at@end@of@kframe{\end{minipage}}%
  \begin{minipage}{\columnwidth}%
 \fi\fi%
 \def\FrameCommand##1{\hskip\@totalleftmargin \hskip-\fboxsep
 \colorbox{shadecolor}{##1}\hskip-\fboxsep
     % There is no \\@totalrightmargin, so:
     \hskip-\linewidth \hskip-\@totalleftmargin \hskip\columnwidth}%
 \MakeFramed {\advance\hsize-\width
   \@totalleftmargin\z@ \linewidth\hsize
   \@setminipage}}%
 {\par\unskip\endMakeFramed%
 \at@end@of@kframe}
\makeatother

\definecolor{shadecolor}{rgb}{.97, .97, .97}
\definecolor{messagecolor}{rgb}{0, 0, 0}
\definecolor{warningcolor}{rgb}{1, 0, 1}
\definecolor{errorcolor}{rgb}{1, 0, 0}
\newenvironment{knitrout}{}{} % an empty environment to be redefined in TeX

\usepackage{alltt}
\usepackage{fontspec}
\setsansfont[Mapping=tex-text]{Fira Sans}
\setcounter{secnumdepth}{4}
\setcounter{tocdepth}{4}
\usepackage[normalem]{ulem}
\usepackage[T1]{fontenc}
\usepackage{dcolumn}
\usepackage{booktabs}
\usepackage{setspace}
\makeatletter
\usetheme{metropolis}
\setbeamertemplate{frame footer}{Bosancianu | Schaub | Hertie School}
\setbeamerfont{page number in head/foot}{size=\tiny}
\setbeamercolor{footline}{fg=gray}
\usepackage{xcolor}
\usepackage{tikz}
\usetikzlibrary{arrows, positioning}
\usepackage[labelformat=empty]{caption}
% For table captions in Beamer
\usepackage[sectionbib]{apacite}
\renewcommand{\bibliographytypesize}{\footnotesize}
\makeatletter
\let\st@rtbibsection\@bibnewpage
\let\st@rtbibchapter\@bibnewpage
\makeatother
\usepackage{amsmath, mathtools}
\usepackage{xunicode}
\usepackage{hyperref}
\graphicspath{{./figures/}} 
% Defines a checkmark
\def\checkmark{\tikz\fill[scale=0.4,color=orange](0,.35) -- (.25,0) -- (1,.7) -- (.25,.15) -- cycle;}
% boxes
\def\boxitorange#1{%
  \smash{\color{orange}\fboxrule=1pt\relax\fboxsep=2pt\relax%
  \llap{\rlap{\fbox{\vphantom{0}\makebox[#1]{}}}~}}\ignorespaces
}
\def\boxitblue#1{%
  \smash{\color{blue}\fboxrule=1pt\relax\fboxsep=2pt\relax%
  \llap{\rlap{\fbox{\vphantom{0}\makebox[#1]{}}}~}}\ignorespaces
}

\newcommand{\indep}{\perp \!\!\!\! \perp}
\setbeamertemplate{itemize items}{\checkmark}
\usepackage{multirow}
\hypersetup{pdfauthor={Bosancianu and Schaub},
	pdftitle={Statistical Modeling and Causal Inference with R},
	pdfsubject={Week 2: Potential Outcomes Framework},
	pdfkeywords={Berlin, Hertie, 2020, week 2}}
\title{\textsc{Statistical Modeling and Causal Inference with R}}
\subtitle{Week 2: Potential Outcomes Framework}
\date{September 14, 2020}
\author{Manuel Bosancianu \hfill Max Schaub}
\institute{Hertie School of Governance}
\IfFileExists{upquote.sty}{\usepackage{upquote}}{}
\begin{document}
\maketitle



\begin{frame}
	\frametitle{Today's focus}
	
	\begin{itemize}
		\setlength\itemsep{1.5em}
		\item Potential outcomes framework (POF)
		\item Biases in causal inference
		\item Assumptions underpinning POF
		\item Randomization (RCTs) in POF
	\end{itemize}
\end{frame}


\begin{frame}
	\frametitle{Last week}
	
	In public policy, we are frequently interested in the causal effect* of an intervention (or phenomenon) on an important outcome.
	
	\begin{figure}
		\centering
		\scriptsize
		\begin{tikzpicture}[->,
			>=stealth',
			shorten >=1pt,
			auto,
			node distance=5cm,
			thick]
			
			\tikzstyle{round}=[circle, thick, auto, draw, minimum size=6mm]
			\tikzstyle{square}=[rectangle, thick, auto, draw, minimum size=6mm]
					
			\node[draw=none, align=center] (1) {Inter-group contact (\texttt{X})};
			\node[draw=none, align=center] (2) [right of=1] {Prejudice (\texttt{Y})};
			\path[every node/.style={font=\sffamily\small}] (1) edge node {} (2);
			
		\end{tikzpicture}
	\end{figure}
	\pause
	
	Assuming we have data to answer this, it's easy to compute the magnitude of association.\bigskip
	

	


	Example from France in 2014 (with ESS data): $-0.14$; the ``contact hypothesis'' \cite{allport_nature_1954} receives support.
	
\end{frame}



\begin{frame}
	\frametitle{Last week}
	
	Difficulties stem from the multiple dynamics that generate an association between the two. Naturally, $contact \Rightarrow prejudice$.
	
	What \textcolor{orange}{other} reasons might there be for an association?\pause
	
    \begin{figure}[ht]
    \scriptsize
        \begin{minipage}[b]{0.45\linewidth}
            \centering
            \begin{tikzpicture}[->, >=stealth', shorten >=1pt, auto, node distance=2.5cm, thick]
      \tikzstyle{round}=[circle, thick, auto, draw, minimum size=6mm]
			\tikzstyle{square}=[rectangle, thick, auto, draw, minimum size=6mm]
					
			\node[round, align=center] (1) {\texttt{Y}};
			\node[round, align=center] (2) [right of=1] {\texttt{X}};
			\path (1) edge node {} (2);
			\end{tikzpicture}
        \end{minipage}
        \hspace{0.3cm}
        \begin{minipage}[b]{0.45\linewidth}
            \centering
            \begin{tikzpicture}[->, >=stealth', shorten >=1pt, auto, node distance=2.5cm, thick]
			\tikzstyle{round}=[circle, thick, auto, draw, minimum size=6mm]
			\tikzstyle{square}=[rectangle, thick, auto, draw, minimum size=6mm]
			\node[round, align=center] (1) {\texttt{X}};
			\node[round, align=center] (2) [right of=1] {\texttt{Y}};
			\node[square, align=center] (3) [above right of=1, xshift=-0.6cm] {\texttt{Z}};
			\path (1) edge node {} (2);
			\path[color=orange] (3) edge node {} (1);
			\path[color=orange] (3) edge node {} (2);
			\end{tikzpicture}
        \end{minipage}
    \end{figure}
	\pause
	
	In this case, $Z$ might be education.
	
\end{frame}


\begin{frame}
	\frametitle{Last week}
	
	Causal inference is primarily a process of \textit{reasoning}:
	
	\begin{itemize}
	  \item eliminating alternative (non-causal) explanations for the association
	  \item making explicit the assumptions needed for a causal relationship
	\end{itemize}\bigskip
	
	It requires an empirical step: marshalling data to show these arguments hold.\bigskip
	
	We rely on a \textit{probabilistic} understanding of causality: causes change the probabilities of their effects.
	
\end{frame}



\section{Potential Outcomes Framework}

\begin{frame}
	\frametitle{Foundations}
	
  The \textcolor{orange}{POF} is a very convenient way to illustrate concepts like \textit{confounding} or \textit{selection bias}.\bigskip
  
  Each individual $i$ in a population can be exposed ($d_i=1$) to a \textit{cause}, or not exposed ($d_i=0$) (e.g., contact with member of different ethnic group).\bigskip
  \pause
  
  For each $i$, assume we can assess an outcome (e.g., prejudice level) \textit{both} after being exposed and not exposed: $y_{d,i}$.
  
  \begin{center}
    \textcolor{orange}{$y_{0,i}$} = non-exposure \hspace{1cm} \textcolor{orange}{$y_{1,i}$} = exposure
  \end{center}

\end{frame}


\begin{frame}
	\frametitle{Notation}
  
  \begin{table}[!ht]
  \centering
  \begin{tabular}{l c}
  \toprule
  \citeA{angrist_mastering_2015} & $y_{0i}$ and $y_{1i}$ \\
  \citeA{cunningham_causal_2021} & $y_i^0$ and $y_i^1$ \\
  \citeA{gerber_field_2012} & $y_i(0)$ and $y_i(1)$ \\
  \bottomrule
  \end{tabular}
  \label{tab:01}
  \caption{Notation differences}
  \end{table}\bigskip
  
  We follow \citeA{angrist_mastering_2015} in notation for the rest of the class.

\end{frame}


\begin{frame}
	\frametitle{Individual treatment effect (ITE)}
  
  \begin{equation}
    ITE = \delta_i= y_{1i} - y_{0i}
  \label{eq:01}
  \end{equation}
  
  The $ITE$ is a ``comparison of two states of the world'' \cite{cunningham_causal_2021}: individual $i$ exposed to contact, and not exposed to it.\bigskip
  \pause

  \textit{Fundamental problem} \cite{holland_statistics_1986} of causal inference: in reality, we only see one of these two states.

\end{frame}


\begin{frame}
  \frametitle{Observable outcomes}
  
  Each observable outcome can be described by a \textit{switching equation}:
  
  \begin{equation}
    y_i = d_iy_{1i} + (1 - d_i)y_{0i}
  \label{eq:02}
  \end{equation}\bigskip
  \pause
  Once $D$ is fixed, we only see \textit{one of} the values: $y_{0i}$ or $y_{1i}$.\bigskip
  
  The $ITE$ is unattainable.
\end{frame}


\begin{frame}
  \frametitle{Average treatment effect (ATE)}
  
  For a population under study, we are typically interested in the ATE:
  
  \begin{equation}
  E[\delta_i] = E[y_{1i} - y_{0i}] = E[y_{1i}] - E[y_{0i}]
    \label{eq:03}
  \end{equation}\bigskip
  \pause
  
  $E[X]$ is the expectation of a random variable, $X$, computed
  
  \begin{equation}
  E[X] = \sum x*p(X=x)
    \label{eq:04}
  \end{equation}
  
  The $ATE$ is unattainable as well.
  
\end{frame}


\begin{frame}
  \frametitle{Example (intermission)}
  
  You are studying a (very small) population of students, assigned to a dorm room with a co-ethnic ($contact=0$) or a peer from a different ethnic group ($contact=1$).
  
  \begin{table}[!ht]
  \centering
  \scriptsize
  \begin{tabular}{c c c}
  \toprule
  \textbf{Student (\textit{i})} & \multicolumn{2}{c}{\textbf{Prejudice}} \\
  & $y_{0i}$ & $y_{1i}$ \\
  \midrule
  1 & 6 & 5 \\
  2 & 4 & 2 \\
  3 & 4 & 4 \\
  4 & 6 & 7 \\
  5 & 3 & 1 \\
  6 & 2 & 2 \\
  7 & 8 & 7 \\
  8 & 4 & 5 \\
  \bottomrule
  \end{tabular}
  \label{tab:02}
  \caption{Potential outcomes for 8 students}
  \end{table}



\end{frame}


\begin{frame}
  \frametitle{Example (intermission)}
  
  \begin{table}[!ht]
  \centering
  \scriptsize
  \begin{tabular}{c c c c}
  \toprule
  \textbf{Student (\textit{i})} & \multicolumn{3}{c}{\textbf{Prejudice}} \\
  & $y_{0i}$ & $y_{1i}$ & \textcolor{orange}{$\delta_i$} \\
  \midrule
  1 & 6 & 5 & -1 \\
  2 & 4 & 2 & -2 \\
  3 & 4 & 4 & 0 \\
  4 & 6 & 7 & 1 \\
  5 & 3 & 1 & -2 \\
  6 & 2 & 2 & 0 \\
  7 & 8 & 7 & -1 \\
  8 & 4 & 5 & 1 \\
  \bottomrule
  \end{tabular}
  \label{tab:03}
  \caption{ITEs}
  \end{table}
  
  \begin{equation}
  \footnotesize
  ATE = E[\delta_i] = \frac{-1+(-2)+0+1+(-2)+0+(-1)+1}{8} = -0.5
    \label{eq:05}
  \end{equation}
  
\end{frame}



\begin{frame}
  \frametitle{Example (intermission)}
  
  Imagine we now also know the treatment assignment of our subjects.

\begin{table}[!ht]
  \centering
  \scriptsize
  \begin{tabular}{c c c c c}
  \toprule
  \textbf{Student (\textit{i})} & \multicolumn{3}{c}{\textbf{Prejudice}} & \textbf{Contact} \\
  & $y_{0i}$ & $y_{1i}$ & \textcolor{orange}{$\delta_i$} &  \\
  \midrule
  1 & 6 & 5 & -1 & 0 \\
  2 & 4 & 2 & -2 & 1 \\
  3 & 4 & 4 & 0 & 0\\
  4 & 6 & 7 & 1 & 0\\
  5 & 3 & 1 & -2 & 1 \\
  6 & 2 & 2 & 0 & 1\\
  7 & 8 & 7 & -1 & 0 \\
  8 & 4 & 5 & 1 & 0\\
  \bottomrule
  \end{tabular}
  \label{tab:04}
  \caption{With treatment assignment}
  \end{table}

\end{frame}


\begin{frame}
	\frametitle{Two more quantities: ATT and\dots}
  
  \textit{Average treatment effect among the treated} (ATT):
  
  \begin{align}
  \footnotesize
    ATT &= E[\delta_i|d_i=1] \nonumber \\
        &= E[y_{1i} - y_{0i}|d_i=1] \\
        &= E[y_{1i}|d_i=1] - E[y_{0i}|d_i=1] \nonumber 
  \label{eq:06}
  \end{align}
  
  \textcolor{orange}{What is the ATT in our example?}
  \pause
  
  \begin{equation}
  \footnotesize
  ATT = \frac{-2+(-2)+0}{3} = -1.333
    \label{eq:07}
  \end{equation}

\end{frame}


\begin{frame}
  \frametitle{Two more quantities: \dots ATU}
  
  \textit{Average treatment effect among the untreated} (ATU):
  
  \begin{align}
  \footnotesize
    ATU &= E[\delta_i|d_i=0] \nonumber \\
        &= E[y_{1i} - y_{0i}|d_i=0] \\
        &= E[y_{1i}|d_i=0] - E[y_{0i}|d_i=0] \nonumber 
  \label{eq:08}
  \end{align}
  
  \textcolor{orange}{What is the ATU in our example?}
  \pause
  
  \begin{equation}
  \footnotesize
  ATU = \frac{-1+0+1+(-1)+1}{5} = 0
    \label{eq:09}
  \end{equation}
  
\end{frame}


\begin{frame}
  \frametitle{An attainable quantity: NATE}
  
  All of the quantities above depend on us having perfect information about alternative states of the world.\bigskip
  
  \begin{table}[!ht]
  \centering
  \scriptsize
  \begin{tabular}{c c c c c}
  \toprule
  \textbf{Student (\textit{i})} & \multicolumn{3}{c}{\textbf{Prejudice}} & \textbf{Contact} \\
  & $y_{0i}$ & $y_{1i}$ & \textcolor{orange}{$\delta_i$} &  \\
  \midrule
  1 & 6 &   &  & 0 \\
  2 &   & 2 &  & 1 \\
  3 & 4 &   &  & 0\\
  4 & 6 &   &  & 0\\
  5 &   & 1 &  & 1 \\
  6 &   & 2 &  & 1\\
  7 & 8 &   &  & 0 \\
  8 & 4 &   &  & 0\\
  \bottomrule
  \end{tabular}
  \label{tab:05}
  \caption{Information we \textit{do} have}
  \end{table}
  
  
\end{frame}


\begin{frame}
  \frametitle{An attainable quantity: NATE}
  
  This allows us to compute a \textit{na\"{i}ve} estimate of the $ATE$:
  
  \begin{align}
  \footnotesize
    NATE &= E[y_{1i}|d_i=1] - E[y_{0i}|d_i=0] \nonumber \\
         &= \frac{2+1+2}{3} - \frac{6+4+6+8+4}{5} \\
         &= 1.666 - 5.6 \nonumber \\
         &= -3.933 \nonumber 
  \label{eq:10}
  \end{align}
  
  Notice that in this case $NATE \neq ATE$.
  
\end{frame}


\begin{frame}
  \frametitle{Connections: ATE--ATT--ATU}
  
  Having perfect information allows us to understand the connection between $ATE$, $ATT$, and $ATU$.
  
  \begin{equation}
  \footnotesize
  ATE = p*ATT + (1-p)*ATU,\;where\;p=prob(D=1)
    \label{eq:12}
  \end{equation}
  \pause
  In our example of inter-ethnic contact:
  
  \begin{align}
  \footnotesize
    ATE &= \frac{3}{8}(-1.333) - \frac{5}{8}0 \nonumber \\
        &= -0.5 - 0 \\
        &= -0.5 \nonumber
  \label{eq:13}
  \end{align}
  
\end{frame}


\begin{frame}
  \frametitle{Connections: NATE--ATE--ATT--ATU}
  
  A bit of mathematical derivation \cite[p.~90]{cunningham_causal_2021} lets us see how these are linked to the $NATE$.
  
  \begin{equation}
  \footnotesize
  NATE = ATE + E[Y_0|D=1] - E[Y_0|D=0] + (1-p)(ATT-ATU)
  \label{eq:13}
  \end{equation}
  \pause
  The formula tells us how something we settle for ($NATE$) in actual analyses is related to something we strive for ($ATE$).
\end{frame}



\section{Defining biases}

\begin{frame}
  \frametitle{Two sources of bias}
  
  \begin{equation}
  \footnotesize
  NATE = ATE + \underbrace{E[Y_0|D=1] - E[Y_0|D=0]}_{selection\;bias} + \underbrace{(1-p)(ATT-ATU)}_{HTE\;bias}
  \end{equation}
  
  $HTE$ = \textit{heterogeneous treatment effect}\bigskip
  \pause
  
  \textcolor{orange}{Selection bias}: the difference in expected outcomes in the absence of treatment for the actual treatment and control group.
\end{frame}

\begin{frame}
  \frametitle{Selection bias}
  
  \begin{table}[!ht]
  \centering
  \scriptsize
  \begin{tabular}{c c c c c}
  \toprule
  \textbf{Student (\textit{i})} & \multicolumn{3}{c}{\textbf{Prejudice}} & \textbf{Contact} \\
  & $y_{0i}$ & $y_{1i}$ & \textcolor{orange}{$\delta_i$} &  \\
  \midrule
  1 & \boxitblue{0.15cm}6 & 5 & -1 & 0 \\
  2 & \boxitorange{0.15cm}4 & 2 & -2 & 1 \\
  3 & \boxitblue{0.15cm}4 & 4 & 0 & 0\\
  4 & \boxitblue{0.15cm}6 & 7 & 1 & 0\\
  5 & \boxitorange{0.15cm}3 & 1 & -2 & 1 \\
  6 & \boxitorange{0.15cm}2 & 2 & 0 & 1\\
  7 & \boxitblue{0.15cm}8 & 7 & -1 & 0 \\
  8 & \boxitblue{0.15cm}4 & 5 & 1 & 0\\
  \bottomrule
  \end{tabular}
  \end{table}
  
  \begin{align}
  \scriptsize
    Bias &= \frac{4+3+2}{3} - \frac{6+4+6+8+4}{5} \nonumber \\
         &= 3 - 5.6 = -2.6
  \label{eq:14}
  \end{align}
  
\end{frame}



\begin{frame}
  \frametitle{HTE bias}
  
  \textcolor{orange}{HTE bias}: the difference in \textit{returns to treatment} (the treatment effect) between the treatment and control group, multiplied by the share of the population in control.\bigskip
  
  \begin{table}[!ht]
  \centering
  \scriptsize
  \begin{tabular}{c c c c c}
  \toprule
  \textbf{Student (\textit{i})} & \multicolumn{3}{c}{\textbf{Prejudice}} & \textbf{Contact} \\
  & $y_{0i}$ & $y_{1i}$ & \textcolor{orange}{$\delta_i$} &  \\
  \midrule
  1 & 6 & 5 & \boxitblue{0.15cm}-1 & 0 \\
  2 & 4 & 2 & \boxitorange{0.15cm}-2 & 1 \\
  3 & 4 & 4 & \boxitblue{0.15cm}0 & 0\\
  4 & 6 & 7 & \boxitblue{0.15cm}1 & 0\\
  5 & 3 & 1 & \boxitorange{0.15cm}-2 & 1 \\
  6 & 2 & 2 & \boxitorange{0.15cm}0 & 1\\
  7 & 8 & 7 & \boxitblue{0.15cm}-1 & 0 \\
  8 & 4 & 5 & \boxitblue{0.15cm}1 & 0\\
  \bottomrule
  \end{tabular}
  \end{table}

\end{frame}



\begin{frame}
  \frametitle{HTE bias}
  
  \begin{align}
  \footnotesize
    Bias &= \frac{5}{8}(\frac{-2+(-2)+0}{3} - \frac{-1+0+1+(-1)+1}{5}) \nonumber \\
         &= \frac{5}{8}(-1.333 - 0) = -0.833
  \label{eq:15}
  \end{align}
  \pause
  Does it all add up?
  
  \begin{align}
  \footnotesize
    NATE   &= ATE + Selection\;bias + HTE\;bias \nonumber \\
    -3.933 &= -0.5 + (-2.6) + (-0.833) \\
    -3.933 &= -3.933 \nonumber
  \label{eq:16}
  \end{align}

\end{frame}


\begin{frame}
  \frametitle{Importance of biases}
  
  \begin{equation}
    NATE = ATE + Selection\;bias + HTE\;bias
  \end{equation}
  
  In our empirical work, we can't directly compute $ATE$, only $NATE$.\bigskip
  \pause
  
  Using the POF, we have shown, though, that $NATE$ is a good substitute for $ATE$ only in the absence of selection bias and HTE bias.
  
\end{frame}


\begin{frame}
  \frametitle{Importance of biases}
  
  \begin{table}[!ht]
  \centering
  \scriptsize
  \begin{tabular}{c c c c c}
  \toprule
  \textbf{Student (\textit{i})} & \multicolumn{3}{c}{\textbf{Prejudice}} & \textbf{Contact} \\
  & $y_{0i}$ & $y_{1i}$ & \textcolor{orange}{$\delta_i$} &  \\
  \midrule
  1 & 6 & 5 & -1 & 0 \\
  2 & 4 & 2 & -2 & 1 \\
  3 & 4 & 4 & 0 & 0\\
  4 & 6 & 7 & 1 & 0\\
  5 & 3 & 1 & -2 & 1 \\
  6 & 2 & 2 & 0 & 1\\
  7 & 8 & 7 & -1 & 0 \\
  8 & 4 & 5 & 1 & 0\\
  \bottomrule
  \end{tabular}
  \end{table}
  
  Here, biases appeared because those who had inter-ethnic contact had lower levels of prejudice (as would be expected if individuals self-select into contact).
  
\end{frame}

\begin{frame}
  \frametitle{Tackling biases}
  
  If selection bias is due to differences in observables, it can easily be tackled.\bigskip
  
  If the bias is due to education, we can control for it statistically: contrast treatment and control groups with identical education level.\bigskip
  
  \pause
  
  A bigger challenge is in how to control for all possible factors, some of which we might not be aware of.
  
\end{frame}



\section{Assumptions of causal identification}

\begin{frame}
  \frametitle{SUTVA: stable unit-treatment value assumption}
  
  Also called \textit{non-interference} by \citeA{gerber_field_2012}.\bigskip
  
  As its name says, it requires that the ``treatment value'' is the same across all units in the population:
  
  \begin{enumerate}
    \item the treatment is of uniform intensity across units*
    \item no externalities: one subject's potential outcome is not affected by the treatment assignment of other subjects (both in terms of mechanism and treatment effect)
  \end{enumerate}\bigskip
  \pause
  
  In our student dorm room contact experiment, how would SUTVA be violated?
  
\end{frame}


\begin{frame}
  \frametitle{Excludability}
  
  The only reason for the change in potential outcomes is the treatment.\bigskip
  
  In our contact experiment, this would be violated if other agents (parents, campus organizations) focus efforts more on the control group.\bigskip
  \pause
  
  It also breaks down when we introduce measurement asymmetries: e.g. more skilled enumerators to measure prejudice in treatment group.
  
\end{frame}


\section{Value of random assignment}

\begin{frame}
  \frametitle{Random assignment}
  
  There are a few types (simple, complete, cluster, block), but we will discuss simple/complete sampling.\bigskip
  
  Defining feature: $p(d_i=1)$ is identical for all subjects. If population is $N$ and $m$ are assigned to treatment, $p = \frac{m}{N}$.\bigskip
  \pause
  
  Under such conditions, treatment status is independent of potential outcomes, and all background attributes, $\boldsymbol{X}$ (\textit{ignorability}).
  
  \begin{equation}
  Y_{0i}, Y_{1i}, \boldsymbol{X} \indep D_i
    \label{eq:16}
  \end{equation}
  
\end{frame}


\begin{frame}
  \frametitle{Mechanics of random assignment (RA)}
  
  Randomly select $m$ individuals out of $N$ (for our student contact study) in treatment.\bigskip
  
  Therefore, potential outcomes for this group of $m$ are the same (in expectation) as that for the population of $N$.
  
  \begin{equation}
  \footnotesize
  E[Y_{1i}|D_i=1] = E[Y_{1i}]
    \label{eq:17}
  \end{equation}
  \pause
  
  The same is the case for the $N-m$ allocated to control group:
  
  \begin{equation}
  \footnotesize
  E[Y_{1i}|D_i=0] = E[Y_{1i}]
    \label{eq:18}
  \end{equation}
  
\end{frame}

\begin{frame}
  \frametitle{Mechanics of random assignment (RA)}
  
  Putting Eq. \ref{eq:17} and \ref{eq:18} together reveals that under RA, treatment and control groups have the same expected potential outcome.
  
  \begin{equation}
  \footnotesize
  E[Y_{1i}|D_i=1] = E[Y_{1i}|D_i=0]
    \label{eq:19}
  \end{equation}
  \pause
  We also have
  
  \begin{equation}
  \footnotesize
  E[Y_{0i}|D_i=1] = E[Y_{0i}|D_i=0]
    \label{eq:20}
  \end{equation}
  
\end{frame}


\begin{frame}
  \frametitle{Mechanics of random assignment (RA)}
  
  We defined $ATE$ as
  
  \begin{equation}
  ATE = E[Y_{1i}] - E[Y_{0i}]
    \label{eq:21}
  \end{equation}
  
  With RA (under Eq. \ref{eq:17} and a similar one for $E[Y_{0i}]$), we can re-write the $ATE$ as
  
  \begin{equation}
  ATE = E[Y_{1i}|D_i=1] - E[Y_{0i}|D_i=0]
    \label{eq:22}
  \end{equation}
  \pause
  The two expectations in Eq. \ref{eq:22} are quantities we can observe in real life ($NATE=ATE$)!
  
\end{frame}


\begin{frame}
  \frametitle{RA as bias killer}
  
  \begin{equation}
  \footnotesize
  NATE = ATE + \underbrace{E[Y_0|D=1] - E[Y_0|D=0]}_{selection\;bias} + \underbrace{(1-p)(ATT-ATU)}_{HTE\;bias}
  \end{equation}
  
  With RA, we know that $E[Y_{0i}|D_i=1] = E[Y_{0i}|D_i=0]$ (Eq. \ref{eq:20}): \textit{selection bias} disappears.
  
  \pause
  
  \begin{equation}
    \begin{cases}
    ATT = E[Y_{1i}|D_i=1] - E[Y_{0i}|D_i=1] \\
    ATU = E[Y_{1i}|D_i=0] - E[Y_{0i}|D_i=0] 
    \end{cases}
    \label{eq:23}
  \end{equation}
  
\end{frame}


\begin{frame}
  \frametitle{RA as bias killer}
  
  Based on Eq. \ref{eq:23}
  
  \begin{scriptsize}
    \begin{align}
    ATT - ATU &= E[Y_{1i}|D_i=1] - E[Y_{0i}|D_i=1] - (E[Y_{1i}|D_i=0] - E[Y_{0i}|D_i=0]) \nonumber \\
              &= \underbrace{E[Y_{1i}|D_i=1] - E[Y_{1i}|D_i=0]}_{=0 (Eq. \ref{eq:19})} + \underbrace{E[Y_{0i}|D_i=0] - E[Y_{0i}|D_i=1]}_{=0 (Eq. \ref{eq:20})}
    \label{eq:24}
    \end{align}
\end{scriptsize}

HTE bias is also addressed by mechanics of RA.\bigskip
\pause

\begin{equation}
  \footnotesize
  NATE = ATE + \underbrace{E[Y_0|D=1] - E[Y_0|D=0]}_{=0} + \underbrace{(1-p)(ATT-ATU)}_{=0}
  \end{equation}
  
\end{frame}

\begin{frame}
  \frametitle{NATE \textit{unbiased} under RA}
  
  In this sense, under RA, NATE is an unbiased estimator of the ATE.\bigskip
  
  If $\theta$ is the parameter we want to estimate, such as the ATE, and $\hat{\theta}$ is an estimator for this, such as the NATE\dots
  
  \begin{equation}
    \hat{\theta}\;is\;unbiased\; if\;E[\hat{\theta}] = \theta
  \end{equation}\bigskip
  
  \pause
  
  $E[\hat{\theta}]$ is the average estimate we would get if we apply the estimator to all possible realizations of the experiment or observational study.
  
\end{frame}


\section{Recap}

\begin{frame}
  \frametitle{Conclusions}
  
  The POF gives us:
  
  \begin{itemize}
    \item the notation to clearly express important concepts like \textit{causal effect} or \textit{bias}
    \item a world of perfect information, to illustrate the connections between these concepts
  \end{itemize}\bigskip
  \pause
  
  The real world falls short of this; we want $ATE$, but have to settle for $NATE$.\bigskip
  
  \begin{equation}
    NATE = ATE + Selection\;bias + HTE\;bias
  \end{equation}
  
\end{frame}


\begin{frame}
  \frametitle{Conclusions}
  
  Random assignment is a situation where $NATE = ATE$, by eliminating bias.\bigskip
  \pause
  
  If properly implemented, treatment and control groups should be similar in everything except the \textit{treatment} itself.\bigskip
  \pause
  
  Regression tries to achieve the same outcome, by controlling for all possible relevant differences between treatment and control groups...
  
\end{frame}



% END
\begin{frame}
\begin{center}
    \Huge Thank \textcolor{orange}{you} for the kind attention!
\end{center}
\end{frame}

% REFERENCES %

\begin{frame}
\frametitle{References}
\bibliographystyle{apacite}
\bibliography{../Bibliography}
\vspace{5cm}
\end{frame}

\end{document}
